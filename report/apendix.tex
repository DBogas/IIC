\documentclass[12pt]{article}
\title{Apendix for STCP network study report}
\author{Diogo Bogas}
\date{12 December 2016}

\begin{document}
\maketitle

\section{Data Structures}

In this particular section, all the data structures used while crunching the data together will be described by chronological order.
the first piece of information retrieved from the STCP website, was about the lines, so the structure BusLine follows:
\begin{itemize}
\item sentido
\item accessibility
\item code
\item description
\item pubcode
\item LineStops
\end{itemize}
Sentido is an integer that can take the values 0 or 1 and represents the direction a bus goes through a line. For example, the 207 line can go from Campanhã to Mercado da Foz, and from Mercado da Foz to Campanhã.\\
Accessibility is an integer that dictates the what special access conditions a bus line has. It can have wheel chair access, baby stroler access for example. Its a variable that was born when the information was beeing gathered, but ended up remaining in the structure.\\
Code is the server way to tell bus lines from each other.\\
Description is a String that tells us what appears in front of a bus that if going through a line.\\
PubCode is the people's way to tell lines from one another.\\
And lineStops is a List of Strings that represents the Stops that compose a line.\\

After knowing each line's composing Stops, a list of all the stops was constructed, and the info for each stop was stored locally. The JAVA structure to represent a Stop is:

\begin{enumerate}
	\item stopCode, the ID of a Stop.
	\item address.
	\item zone, the part of the city a Stop belongs to.
	\item name, the way people know which stop is which.
	\item longitude.
	\item latitude.
	\item totalLinesServed, a number.
	\item linesServed, a list of the lines served.
	\item adjacentStops, the stops it can connect to.
	\item distanceBFS, an auxiliary variable to a BFS method.
\end{enumerate}
\newpage
	The last four variables are derived, and are used as auxiliary variables to some methods. In a graph, a Stop is a node. The edges are the interactions between nodes:
	
\begin{enumerate}
	\item source;
	\item target;
	\item desc;
	\item weight;
\end{enumerate}

Source and Target define the edge, as the network is a directed graph. Desc is a String composed by both stopCodes, source + - + target.
The weight variable represents the number of lines that use that edge. 


When grouping the information by street, the nodes of the graph(streets themselves) were represented by the BusStreet structure:

\begin{itemize}
	\item street
	\item stops, a list of all the stops in that street
	\item  neighbours
	\item longitude;
	\item latitude;
\end{itemize}

	The neighbours attribute was supposed to represent the streets adjacent to a particular street, but ended up not beeing used. The edges in this type of grouping are reresented by AdressEdge:
	
	\begin{itemize}
	\item src
	\item dest
	\item weight
	\item nome
	\end{itemize}
	
	The graph is directed still, weight is the number of lines that use that particular interaction and nome is a way to diferentiate the edges, and is derived from the source and target streets.
	
	The final way to group the stops is by code, so the Spot edge was created:
	
	\begin{itemize}
	\item code
	\item stops
	\item LinesServed
	\end{itemize}
	
Code is the way to tell Spots apart, and it's derivarion is described in the main document.\\
Stops is a list of stops the Spot has in itself.\\
LinesServed is a list with all the lines that go through that Spot.\\

The edges in this type of grouping are represented by the SpotEdge structure:
\begin{itemize}
\item from
\item to
\item weight
\item name
\end{itemize}

From and to are the source and target Spots, weight and name, as the other types of edges are the number of lines that use the edge and a way to distinguish the edges respectively.

This ends this appendix.

\end{document}